% ---------- functions.tex ----------
%
% Public, document-facing commands used by the resume sections.
% Commands with a '*' suffix automatically append a line break.
%
% Best practices:
% - All commands use expl3 (LaTeX3) for robust text processing.
% - Comma-separated lists are parsed and trimmed automatically.
% - Empty/blank items are filtered out to prevent formatting errors.
% - Uses dedicated variables with \l__cv_ prefix to avoid naming conflicts.

% ---- contact info line ----
% Usage:
% \contactinfo{ \contactentry{text}{url}, \contactentry{text}{url}, ... }
% Output: entry1 | entry2 | entry3   (with consistent spacing)
%
% Note: Automatically filters empty entries and trims whitespace.
\ExplSyntaxOn

\seq_new:N \l__cv_contact_seq
\tl_new:N  \l__cv_contact_item_tl

\NewDocumentCommand{\contactentry}{m m}{%
  \linkplain{#1}{#2}%
}

\NewDocumentCommand{\contactinfo}{m}{
  \__cv_contactinfo:n { #1 }
}

\cs_new_protected:Npn \__cv_contactinfo:n #1 {
  \seq_clear:N \l__cv_contact_seq

  % Split on commas, trim items, drop empties
  \clist_map_inline:nn { #1 } {
    \tl_set:Nn \l__cv_contact_item_tl { ##1 }
    \tl_trim_spaces:N \l__cv_contact_item_tl

    \tl_if_blank:VTF \l__cv_contact_item_tl
      { }
      { \seq_put_right:NV \l__cv_contact_seq \l__cv_contact_item_tl }
  }

  % Join with a consistent separator and spacing
  \seq_use:Nn \l__cv_contact_seq { ~|~ }
}
\ExplSyntaxOff

% ---- job entry ----
% \job*[seniority]{profession}{role}{Company}{City, ST}{YYYY/MM/DD}{YYYY/MM/DD|present}
% '*' adds a line break at the end.
%
% Parameters:
%   [seniority] - Optional prefix like "Senior" or "Lead"
%   {profession} - Job title, e.g., "Software Engineer"
%   {role} - Specific role/team, can be empty
%   {Company} - Company name
%   {City, ST} - Location
%   {start date} - In YYYY/MM/DD format
%   {end date} - In YYYY/MM/DD format or "present"
\ExplSyntaxOn
\NewDocumentCommand{\job}{s o m m m m m m}{
  \textbf{\IfValueTF{#2}{#2~#3}{#3}}%
  \tl_if_blank:nTF {#4}{}{,~#4}~%
  @~#5~--~#6 \hfill \dateMY{#7}~--~\dateMY{#8}%
  \IfBooleanT{#1}{\\}%
}
\ExplSyntaxOff

% ---- credential id ----
% \id{credential-id}[url]
% Displays "ID credential-id" with an optional hyperlink.
\NewDocumentCommand{\id}{m o}{
  ID~\IfValueTF{#2}{\linkplain{#1}{#2}}{#1}%
}

% ---- education entry ----
% \education*[id][url]{Issuer}{Title}{YYYY/MM/DD}
% '*' adds a line break at the end.
%
% Parameters:
%   [id] - Optional credential/certificate ID
%   [url] - Optional verification URL (requires id)
%   {Issuer} - Institution/organization name
%   {Title} - Degree/certificate title
%   {date} - Completion date in YYYY/MM/DD format
\NewDocumentCommand{\education}{s o o m m m}{
  \textbf{#4}~--~#5%
  \IfValueT{#2}{~(\IfValueTF{#3}{\id{#2}[#3]}{\id{#2}})}%
  \hfill \dateMY{#6}%
  \IfBooleanT{#1}{\\}%
}

% ---- certification entry ----
% \certification*[id][url]{Issuer}{Title}{YYYY/MM/DD}
% '*' adds a line break at the end.
% Note: Automatically prefixes title with "Certification:"
\NewDocumentCommand{\certification}{s o o m m m}{
  \education[#2][#3]{#4}{Certification:~#5}{#6}%
  \IfBooleanT{#1}{\\}%
}

% ---- license entry ----
% \license*[id][url]{Issuer}{Title}{YYYY/MM/DD}
% '*' adds a line break at the end.
% Note: Automatically prefixes title with "License:"
\NewDocumentCommand{\license}{s o o m m m}{
  \education[#2][#3]{#4}{License:~#5}{#6}%
  \IfBooleanT{#1}{\\}%
}

% ---- skill entries ----
% \skill*{Category}{skill1, skill2, ..., skillN}
% '*' adds a line break at the end.
%
% Parameters:
%   {Category} - Skill category name (e.g., "Programming Languages")
%   {skills} - Comma-separated list of skills
%
% Note: Empty items are filtered out, and whitespace is trimmed.
\ExplSyntaxOn

\seq_new:N \l__cv_skill_seq
\tl_new:N  \l__cv_skill_item_tl

\NewDocumentCommand{\skill}{s m m}{
  \__cv_skill_render:nn { #2 } { #3 }
  \IfBooleanT{#1}{\\}%
}

\cs_new_protected:Npn \__cv_skill_render:nn #1 #2 {
  \seq_clear:N \l__cv_skill_seq

  % Split list on commas, trim each item, drop empties
  \clist_map_inline:nn { #2 } {
    \tl_set:Nn \l__cv_skill_item_tl { ##1 }
    \tl_trim_spaces:N \l__cv_skill_item_tl

    \tl_if_blank:VTF \l__cv_skill_item_tl
      { } % blank -> ignore
      { \seq_put_right:NV \l__cv_skill_seq \l__cv_skill_item_tl }
  }

  % Only print if at least one item exists
  \seq_if_empty:NF \l__cv_skill_seq
    { \textbf{#1:}~\seq_use:Nn \l__cv_skill_seq { ,~ } }
}
\ExplSyntaxOff
