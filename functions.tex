% ---------- functions ----------
%
% Public, document-facing commands used by the resume sections.
% All commands handle their own line breaks and vertical spacing.
%
% Best practices:
% - All commands use expl3 (LaTeX3) syntax for robust text processing
% - Comma-separated lists are properly parsed and trimmed
% - Empty/blank items are automatically filtered out
% - Uses dedicated variables with \l__cv_ prefix to avoid naming conflicts

% ---- contact info line ----
% Usage:
% \contactinfo{ \contactentry{text}{url}, \contactentry{text}{url}, ... }
% Output: entry1 | entry2 | entry3   (with consistent spacing)
%
% Note: Automatically filters empty entries and trims whitespace
\ExplSyntaxOn

\seq_new:N \l__cv_contact_seq
\tl_new:N  \l__cv_contact_item_tl

\NewDocumentCommand{\contactentry}{m m}{%
  \linkplain{#1}{#2}%
}

\NewDocumentCommand{\contactinfo}{m}
  { \__cv_contactinfo:n { #1 } }

\cs_new_protected:Npn \__cv_contactinfo:n #1
  {
    \seq_clear:N \l__cv_contact_seq

    % Split on commas, trim items, drop empties
    \clist_map_inline:nn { #1 }
      {
        \tl_set:Nn \l__cv_contact_item_tl { ##1 }
        \tl_trim_spaces:N \l__cv_contact_item_tl

        \tl_if_blank:VTF \l__cv_contact_item_tl
          { }
          { \seq_put_right:NV \l__cv_contact_seq \l__cv_contact_item_tl }
      }

    % Join with a consistent separator and spacing
    \seq_use:Nn \l__cv_contact_seq { ~|~ }
  }

\ExplSyntaxOff

% ---- job entry ----
% \job[seniority]{profession}{role}{Company}{City, ST}{start date}{end date}
%
% Parameters:
%   [seniority] - Optional prefix like "Senior" or "Lead"
%   {profession} - Job title, e.g., "Software Engineer"
%   {role} - Specific role/team, can be empty
%   {Company} - Company name
%   {City, ST} - Location
%   {start date} - In YYYY/MM/DD format
%   {end date} - In YYYY/MM/DD format or "present"
%
% Layout:
% Company                                            Location
% Title, Role                                           Dates
%
\ExplSyntaxOn

\NewDocumentCommand{\job}{o m m m m m m}{%
  \par \addvspace{0.8ex}
  \noindent \textbf{#4} \hfill \textbf{#5} \\
  \textit{\IfValueTF{#1}{#1~#2}{#2}} \tl_if_blank:nF {#3}{,~#3} \hfill \textit{\dateMY{#6}~--~\dateMY{#7}} \par
  \addvspace{0.3ex}
}

\ExplSyntaxOff

% ---- credential id ----
% \id{credential-id}
% \id{credential-id}[url]
% Displays "ID credential-id" with optional hyperlink
\NewDocumentCommand{\id}{m o}{%
  ID~\IfValueTF{#2}{\linkplain{#1}{#2}}{#1}%
}

% ---- education entry ----
% \education[credential-id][optional-url]{Issuer}{Title}{date}
%
% Parameters:
%   [credential-id] - Optional credential/certificate ID
%   [url] - Optional verification URL (requires credential-id)
%   {Issuer} - Institution/organization name
%   {Title} - Degree/certificate title
%   {date} - Completion date in YYYY/MM/DD format
%
\NewDocumentCommand{\education}{o o m m m}{%
  \par \addvspace{0.8ex}
  \noindent \textbf{#3} \hfill \textit{\dateMY{#5}} \\
  #4 \IfValueT{#1}{~(\IfValueTF{#2}{\id{#1}[#2]}{\id{#1}})} \par
  \addvspace{0.3ex}
}

% ---- certification entry ----
% \certification[credential-id][optional-url]{Issuer}{Title}{date}
%
\NewDocumentCommand{\certification}{o o m m m}{%
  \education[#1][#2]{#3}{Certification:~#4}{#5}%
}

% ---- license entry ----
% \license[credential-id][optional-url]{Issuer}{Title}{date}
%
\NewDocumentCommand{\license}{o o m m m}{%
  \education[#1][#2]{#3}{License:~#4}{#5}%
}

% ---- skill entries ----
% \skill{Category}{skill1, skill2, ..., skillN}
%
% Parameters:
%   {Category} - Skill category name (e.g., "Programming Languages")
%   {skills} - Comma-separated list of skills
%
\ExplSyntaxOn

\seq_new:N \l__cv_skill_seq
\tl_new:N  \l__cv_skill_item_tl

\NewDocumentCommand{\skill}{m m}
  { \__cv_skill_render:nn { #1 } { #2 } }

\cs_new_protected:Npn \__cv_skill_render:nn #1 #2
  {
    \seq_clear:N \l__cv_skill_seq

    % Split list on commas, trim each item, drop empties
    \clist_map_inline:nn { #2 }
      {
        \tl_set:Nn \l__cv_skill_item_tl { ##1 }
        \tl_trim_spaces:N \l__cv_skill_item_tl

        \tl_if_blank:VTF \l__cv_skill_item_tl
          { }% blank -> ignore
          { \seq_put_right:NV \l__cv_skill_seq \l__cv_skill_item_tl }
      }

    % Only print if at least one item exists
    \seq_if_empty:NF \l__cv_skill_seq
      {
        \par \addvspace{0.4ex}
        \noindent \textbf{#1:}~\seq_use:Nn \l__cv_skill_seq { ,~ } \par
        \addvspace{0.2ex}
      }
  }

\ExplSyntaxOff
