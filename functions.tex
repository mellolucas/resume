% ---------- functions.tex ----------
%
% Public, document-facing commands used by the resume sections.
% Star variants (*) force a line break after the entry.
%
% Best practices:
% - All commands use expl3 (LaTeX3) syntax for robust text processing
% - Comma-separated lists are properly parsed and trimmed
% - Empty/blank items are automatically filtered out
% - Uses dedicated variables with \l__cv_ prefix to avoid naming conflicts
% - All output is plain text (ATS-compatible, no graphics or special formatting)
%

% ---- contact info line ----
% Usage:
% \contactinfo{ \contactentry{text}{url}, \contactentry{text}{url}, ... }
% Output: entry1 | entry2 | entry3   (with consistent spacing)
%
% Note: Automatically filters empty entries and trims whitespace
\ExplSyntaxOn

\seq_new:N \l__cv_contact_seq
\tl_new:N  \l__cv_contact_item_tl

\NewDocumentCommand{\contactentry}{m m}{%
  \linkplain{#1}{#2}%
}

\NewDocumentCommand{\contactinfo}{m}
  { \__cv_contactinfo:n { #1 } }

\cs_new_protected:Npn \__cv_contactinfo:n #1
  {
    \seq_clear:N \l__cv_contact_seq

    % Split on commas, trim items, drop empties
    \clist_map_inline:nn { #1 }
      {
        \tl_set:Nn \l__cv_contact_item_tl { ##1 }
        \tl_trim_spaces:N \l__cv_contact_item_tl

        \tl_if_blank:VTF \l__cv_contact_item_tl
          { }
          { \seq_put_right:NV \l__cv_contact_seq \l__cv_contact_item_tl }
      }

    % Join with a consistent separator and spacing
    \seq_use:Nn \l__cv_contact_seq { ~|~ }
  }

\ExplSyntaxOff

% ---- job entry ----
% \job[seniority]{profession}{role}{Company}{City, ST}{YYYY/MM/DD}{YYYY/MM/DD|present}
% \job*[seniority]{...} - same output, but ends with \\ for one-entry-per-line usage.
%
% Parameters:
%   [seniority] - Optional prefix like "Senior" or "Lead"
%   {profession} - Job title, e.g., "Software Engineer"
%   {role} - Specific role/team, can be empty
%   {Company} - Company name
%   {City, ST} - Location
%   {start date} - In YYYY/MM/DD format
%   {end date} - In YYYY/MM/DD format or "present"
%
\ExplSyntaxOn

\NewDocumentCommand{\job}{s o m m m m m m}{%
  \textbf{\IfValueTF{#2}{#2~#3}{#3}}%
  \tl_if_blank:nTF {#4}{}{,~#4}~%
  @~#5~--~#6 \hfill \dateMY{#7}~--~\dateMY{#8}%
  \IfBooleanT{#1}{\\}%
}

\ExplSyntaxOff

% ---- credential id ----
% \id{credential-id}
% \id{credential-id}[url]
% Displays "ID credential-id" with optional hyperlink
\NewDocumentCommand{\id}{m o}{%
  ID~\IfValueTF{#2}{\linkplain{#1}{#2}}{#1}%
}

% ---- education entry ----
% \education[credential-id][optional-url]{Issuer}{Title}{YYYY/MM/DD}
% \education*[...]{...} - same output, but ends with \\ for one-entry-per-line usage.
%
% Parameters:
%   [credential-id] - Optional credential/certificate ID
%   [url] - Optional verification URL (requires credential-id)
%   {Issuer} - Institution/organization name
%   {Title} - Degree/certificate title
%   {date} - Completion date in YYYY/MM/DD format
%
\NewDocumentCommand{\education}{s o o m m m}{%
  \textbf{#4}~--~#5%
  \IfValueT{#2}{~(\IfValueTF{#3}{\id{#2}[#3]}{\id{#2}})}%
  \hfill \dateMY{#6}%
  \IfBooleanT{#1}{\\}%
}

% ---- certification entry ----
% \certification[credential-id][optional-url]{Issuer}{Title}{YYYY/MM/DD}
% \certification*[...]{...} - same output, but ends with \\ for one-entry-per-line usage.
%
% Note: Automatically prefixes title with "Certification:"
%
\NewDocumentCommand{\certification}{s o o m m m}{%
  \textbf{#4}~--~Certification:~#5%
  \IfValueT{#2}{~(\IfValueTF{#3}{\id{#2}[#3]}{\id{#2}})}%
  \hfill \dateMY{#6}%
  \IfBooleanT{#1}{\\}%
}

% ---- license entry ----
% \license[credential-id][optional-url]{Issuer}{Title}{YYYY/MM/DD}
% \license*[...]{...} - same output, but ends with \\ for one-entry-per-line usage.
%
% Note: Automatically prefixes title with "License:"
%
\NewDocumentCommand{\license}{s o o m m m}{%
  \textbf{#4}~--~License:~#5%
  \IfValueT{#2}{~(\IfValueTF{#3}{\id{#2}[#3]}{\id{#2}})}%
  \hfill \dateMY{#6}%
  \IfBooleanT{#1}{\\}%
}

% ---- skill entries ----
% \skill{Category}{skill1, skill2, ..., skillN}
% \skill*{...} - same output, but ends with \\ for one-entry-per-line usage.
%
% Parameters:
%   {Category} - Skill category name (e.g., "Programming Languages")
%   {skills} - Comma-separated list of skills
%
% Note: Empty items are filtered out, whitespace is trimmed
%
\ExplSyntaxOn

\seq_new:N \l__cv_skill_seq
\tl_new:N  \l__cv_skill_item_tl
\bool_new:N \l__cv_skill_newline_bool

\NewDocumentCommand{\skill}{s m m}{%
  \bool_set:Nn \l__cv_skill_newline_bool { #1 }
  \__cv_skill_render:nn { #2 } { #3 }
}

\cs_new_protected:Npn \__cv_skill_render:nn #1 #2
  {
    \seq_clear:N \l__cv_skill_seq

    % Split list on commas, trim each item, drop empties
    \clist_map_inline:nn { #2 }
      {
        \tl_set:Nn \l__cv_skill_item_tl { ##1 }
        \tl_trim_spaces:N \l__cv_skill_item_tl

        \tl_if_blank:VTF \l__cv_skill_item_tl
          { }% blank -> ignore
          { \seq_put_right:NV \l__cv_skill_seq \l__cv_skill_item_tl }
      }

    % Only print if at least one item exists
    \seq_if_empty:NF \l__cv_skill_seq
      {
        \textbf{#1:}~\seq_use:Nn \l__cv_skill_seq { ,~ }
        \bool_if:NT \l__cv_skill_newline_bool { \\ }
      }
  }

\ExplSyntaxOff
